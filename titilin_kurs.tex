\documentclass[14pt]{extarticle}
\usepackage{fontspec}
\usepackage[russian]{babel}
\setmainfont{Times New Roman}
\usepackage{amssymb}
\usepackage{setspace}
\usepackage{enumitem}
\onehalfspacing
\usepackage{amsmath}
\usepackage{titlesec} 
\usepackage{listings}
\usepackage{booktabs}
\usepackage{xcolor}
\usepackage{graphicx}
\usepackage{tikz}
\usepackage{url}
\urlstyle{same}
\usepackage{longtable}
\usepackage{titlesec}
\usepackage{indentfirst}
\usepackage[backend=biber, style=numeric,sorting=none]{biblatex}
\DeclareFieldFormat{labelnumberwidth}{#1\adddot}
\DeclareFieldFormat{urldate}{(дата обращения\space#1)}
\DeclareFieldFormat{title}{#1\addcolon}
\setlength{\biblabelsep}{5pt}
\addbibresource{bibl.bib}
\bibliography{sample}
\usepackage{csquotes}
% \setcounter{page}{2} 
\setlength{\parindent}{1.25cm}
\usepackage[right=10mm,left=30mm,top=20mm,bottom=20mm]{geometry}
\titleformat{\section} {\normalfont\Large\bfseries}{\thesection. }{0em}{\centering} 
\titleformat{\subsection } {\normalfont\Large\bfseries}{\thesubsection }{0.5em}{\centering} 
\titleformat{\subsubsection } {\normalfont\Large\bfseries}{\thesubsubsection }{0.5em}{\centering} 
\setlist[enumerate,1]{label=\arabic*)}

\definecolor{codegreen}{rgb}{0,0.6,0}
\definecolor{codegray}{rgb}{0.5,0.5,0.5} \definecolor{codepurple}{rgb}{0.58,0,0.82} \definecolor{backcolour}{rgb}{0.95,0.95,0.92} \lstdefinestyle{mystyle}{
	backgroundcolor=\color{backcolour},
	commentstyle=\color{codegreen},
	keywordstyle=\color{magenta},
	numberstyle=\tiny\color{codegray},
	stringstyle=\color{codepurple},
	basicstyle=\ttfamily\footnotesize,
	breakatwhitespace=false,
	breaklines=true,
	captionpos=b,
	showspaces=false,
	showstringspaces=false,
	showtabs=false,
	tabsize=2
}

\lstset{style=mystyle}

\title{}
\author{}
\date{}
\begin{document}
\begin{titlepage}
\begin{tikzpicture}[remember picture, overlay] \node[anchor=north west] at (current page.north west) {\includegraphics[scale=0.35]{logo.jpg}}; \end{tikzpicture} 
    \begin{center}
        МИНОБРНАУКИ РОССИИ \\
        Федеральное государственное бюджетное образовательное учреждение \\
        высшего образования \\
        «САНКТ-ПЕТЕРБУРГСКИЙ ГОСУДАРСТВЕННЫЙ \\
        ЭКОНОМИЧЕСКИЙ УНИВЕРСИТЕТ» \\
        
        \vspace{0.5cm}
        
        Факультет информатики и прикладной математики
        \vspace{0.5cm}\\
        Кафедра прикладной математики и экономико-математических методов
        
        \vspace{0.5cm}
        \vspace*{0.5cm}
        
        \textbf{КУРСОВАЯ РАБОТА}\\
        по дисциплине: \\
        \textbf{«Численные методы»}\\
        \vspace*{0.5cm}
        Тема:
        <<Реализация и сравнение стационарных итерационных методов решения СЛАУ>>\\
    \end{center}
        Направление ~~ 01.03.02 <<Прикладная Математика и Информатика>>\\ 
        Направленность ~~  << Прикладная математика и информатика в экономике
        и управлении>>\\
        Обучающийся~ Титилин Александр Михайлович   \\
        Группа ~~ ПМ-2201 \hspace{6.5cm}
        Подпись  \underline{\hspace{4cm}}
        \vspace{1cm}\\
        Проверила~ Соловьёва Наталья Анатольевна\\
        Должность~ к.ф.-м.н., доцент
        \vspace{0.5cm}\\
        Оценка:\underline{\hspace{5cm}} \hspace{3cm} Дата: \underline{\hspace{3cm}}
        \vspace{0.5cm}\\
        Подпись: \underline{\hspace{3cm}}
        % \vspace{0.5cm}\\
    \begin{center}
        Санкт-Петербург\\
       2024 г.
    \end{center}
\end{titlepage}
\setcounter{page}{2}
    \tableofcontents{}
    \pagebreak{}
    \section*{ВВЕДЕНИЕ}
    \addcontentsline{toc}{section}{ВВЕДЕНИЕ}
    Целью курсовой работы является реализация итерационных
    алгоритмов решения систем линейных уравнений с использованием
    языка программирования Python.\\
    Задачи курсовой работы:
    \begin{enumerate}
        % \item создание алгоритмов, реализующих итерационные методы решения СЛАУ;
        \item реализация алгоритмов решения СЛАУ с помощью итерационных методов;
        \item создание алгоритмов генерации случайных матриц с необходимыми свойствами;
        \item сравнение времени работы и точности алгоритмов между собой и с прямыми методами;
        \item визуализация сравнений алгоритмов.
    \end{enumerate}
    \pagebreak
    \section{ИТЕРАЦИОННЫЕ СТАЦИОНАРНЫЕ ОДНОШАГОВЫЕ МЕТОДЫ РЕШЕНИЯ СЛАУ}
    Пусть дана СЛАУ вида $Ax = b$, где  $A$ квадратная невырожденная матрица порядка $n$,
     $b$ столбец длины $n$,  $x$ неизвестный столбец.

     Обозначим $x^{(i)}$ $i$-е приближение к точному решению СЛАУ.  $T$ -- оператор перехода,
     отображение, которое переводит приближение в следующее
      \[
     x^{(i + 1)} \leftarrow T(x^{(i)})
     .\] 

     Все рассмотренные далее методы будут реализовывать процесс последовательного применения оператора перехода, начиная
     с некоторого начального приближения.
     \subsection{Итерационный метод Якоби}
     Рассмотрим данный оператор перехода
     \[
     T(x^{(k)}) = \begin{pmatrix} 
     T_1(x^{(k)}) \\
     T_2(x^{(k)})\\
     \vdots\\
     T_{n}(x^{(k)})
     \end{pmatrix} 
     ,\] 
     где
     \[
     T_{i}(x^{k}) = \frac{1}{a_{ii}} (b_{i} - \sum_{j \neq i} a_{ij}x^{(k)}_{j}) ,i = 1\dots n
     .\] 
     В листинге \ref{jacobi} представлена реализация данного метода на языке программирования Python.
     \lstinputlisting[language=Python,caption=Реализация метода Якоби. , label={jacobi}]{code/jacobi.py}
     Метод Якоби сходится при любом начальном приближении, если матрица $A$ в СЛАУ  $Ax=b$ 
     имеет диагональное преобладание \cite{book}.
     \subsection{Итерационный метод Гаусса-Зейделя}
     Рассмотрим оператор перехода
     \[
     T_{i}(x^{(k)}) = \frac{1}{a_{11}} (b_{i} - \sum_{j=1}^{i-1} a_{ii} x_{j}^{(k+1)} - \sum_{j=i+1}^{n} a_{ii} x_{j}^{(k)})
     .\] 
     В листинге \ref{gsl} представлена реализация данного метода.
     \lstinputlisting[language=Python,caption=Реализация метода Гаусса-Зейделя., label={gsl}]{code/gauss_seildel.py}
     Метод Гаусса-Зейделя исправляет недостаток метода Якоби, который не использовал уже вычисленные 
     значения приближения $x^{(k+1)}$. 

     Метод Гаусса-Зейделя сходится из любого начального приближения, если матрица $A$ в СЛАУ
     $Ax= b$ имеет диагональное преобладание или является симметричной и положительно определенной \cite{book}.
      \subsection{Методы релаксации}
      Рассмотрим оператор перехода 
      \[
      T_{i}(x^{(k)}) = (1- \omega) x_{i}^{(k)} + \frac{\omega}{a_{ii}}
      (b_{i} - \sum_{j=1}^{i-1}) (b_{i} - \sum_{j=1}^{i-1} a_{ij} x_{j}^{(k+1)} -
      \sum_{j=i+1} a_{ij} x_{j}^{(k)}), ~
      \text{где}~ \omega \in \mathbb{R}
      .\] 
      Число $\omega$ называется параметром релаксации. В листинге \ref{rel} представлена реализация данного метода.
      \lstinputlisting[label={rel},language=Python,caption=Реализация метода релаксации]{code/relaxation.py}
      Для сходимости метода релаксации должно выполнятся неравенство $0<\omega<2$.
      Методы релаксации сходятся из любого начального приближения, если матрица  $A$ в СЛАУ 
      $Ax=b$  положительно определенная \cite{book}. 
\subsection{Метод Ричардсона}
Рассмотрим оператор перехода
\[
    T(x^{(k)}) = (E - \tau A) x^{(k)} + \tau b, ~ \text{где} ~ \tau = \text{const}, \tau \neq 0
.\] 
В листинге \ref{r} представлена реализация данного метода.
\lstinputlisting[label={r},language=Python, caption=Реализация метода Ричарсона]{code/richardson.py}
Если в СЛАУ $Ax = b$ матрица  $A$ симметрична, положительно определена , то последовательность, порожденная методом Ричардсона с параметром
$\tau = \frac{2}{M + \mu}, M>0,\mu>0$, где $M,\mu$ границы спектра матрицы $A$, сходится из любого начального приближения \cite{book}.
\pagebreak
\section{СРАВНЕНИЕ АЛГОРИТМОВ}
\subsection{Особенности реализации алгоритмов}
У выбранного языка программирования есть ряд недостатков. Он очень медленный 
в математических задачах и не имеет встроенной поддержки матриц и функций линейной алгебры.
Для решения этих проблем используются сторонние библиотеки.

Список сторонних библиотек:
\begin{enumerate}
    \item numpy --  содержит реализации векторов и матриц \cite{numpy};
    \item scipy --  содержит реализации функций линейной алгебры \cite{scipy};
    \item numba -- реализует компиляцию, что необходимо для ускорения вычислений \cite{numba};
    \item pandas -- содержит реализацию таблиц, что необходимо для визуализации результатов сравнений \cite{pandas}.
\end{enumerate}
\subsection{Генерация случайных матриц}
Для проведения сравнений необходимы матрицы, состоящие из случайных чисел.
Данные матрицы должны иметь диагональное преобладание или
быть симметричными и положительно определенными.

Рассмотрим функции для генерации случайной матрицы с диагональным преобладанием порядка $n$, представленный в листинге \ref{gen1}.
Данный алгоритм достаточно быстро создает матрицы, итерационные методы в применении
к данным матрицам быстро сходятся.
\lstinputlisting[label={gen1},language=Python,caption=Генерация случайных матриц с диагональным преобладанием.]{code/generate_dominate_row.py}

Рассмотрим функцию для генерации случайной 
симметричной и положительно определенной матрицы, представленную в листинге \ref{gen2}.
Данный алгоритм начинает работать медленно на матрицах порядка больше 1000,
но более быстрые алгоритмы создания таких матриц,
создают матрицы, итерационные методы в применении к которым сходятся 
более медленно.
\lstinputlisting[label={gen2}, language=Python,
caption=Генерация случайных симметричных и положительно определенных матриц.]{code/sym_pos.py}
\subsection{Сравнение алгоритмов решения СЛАУ с матрицей, имеющей диагональное преобладание}\label{diag}
\subsubsection{Сравнение времени работы алгоритмов}
Алгоритм сравнения алгоритмов:
\begin{enumerate}
    \item создаются случайные квадратные матрицы, имеющие диагональное преобладание, порядка $100,200,\dots,5100$;
    \item  создаются случайные векторы размера $100,200,\dots,5100$;
    \item решаются СЛАУ точным методом, запоминаются решения и затраченное время;
    \item СЛАУ решаются методом Якоби, Гаусса-Зейделя и релаксации с параметром $\omega = 0.3,0.5,1.2$, с компиляцией и без, запоминается затраченное время;
\end{enumerate}
Как можно заметить из таблиц \ref{d1} -- \ref{d10}, самым эффективным является метод Гаусса-Зейделя,
для которого компиляция ускоряет код в 4.5 раза.
\begin{longtable}{lrrr}
\label{d1}\\
\caption{Результаты работы метода Якоби без компиляции}\\
\toprule
 & Количество итераций & Затраченное время & Разность времени \\
Порядок матрицы &  &  &  \\
\midrule
\endfirsthead
\toprule
 & Количество итераций & Затраченное время & Разность времени \\
Порядок матрицы &  &  &  \\
\midrule
\endhead
\midrule
\midrule
\endfoot
\bottomrule
\endlastfoot
100 & 52 & 0.0443694591522 & -0.0417437553406 \\
200 & 78 & 0.1281049251556 & -0.1263153553009 \\
300 & 130 & 0.3254456520081 & -0.3212807178497 \\
400 & 76 & 0.2703664302826 & -0.2578797340393 \\
500 & 126 & 0.6120841503143 & -0.5879716873169 \\
600 & 99 & 0.5401477813721 & -0.5048050880432 \\
700 & 97 & 0.6299719810486 & -0.5812337398529 \\
800 & 69 & 0.5371758937836 & -0.4641580581665 \\
900 & 69 & 0.6227357387543 & -0.5164468288422 \\
1000 & 44 & 0.4429130554199 & -0.3038856983185 \\
1100 & 80 & 0.8988883495331 & -0.6883490085602 \\
1200 & 67 & 0.8679387569427 & -0.6290333271027 \\
1300 & 142 & 2.0736641883850 & -1.7661986351013 \\
1400 & 59 & 0.9120345115662 & -0.5189483165741 \\
1500 & 99 & 1.6660103797913 & -1.1691880226135 \\
1600 & 44 & 0.8478009700775 & -0.2669956684113 \\
1700 & 96 & 2.0153887271881 & -1.3348014354706 \\
1800 & 63 & 1.3615150451660 & -0.5486998558044 \\
1900 & 77 & 1.8149716854095 & -0.8589713573456 \\
2000 & 78 & 1.9024038314819 & -0.6513102054596 \\
2100 & 51 & 1.3669714927673 & -0.0047633647919 \\
2200 & 96 & 2.7012000083923 & -1.2116370201111 \\
2300 & 66 & 1.9693181514740 & -0.2522995471954 \\
2400 & 80 & 2.5349364280701 & -0.6316859722137 \\
2500 & 56 & 1.7136464118958 & 0.4483952522278 \\
2600 & 104 & 3.7057044506073 & -1.2810068130493 \\
2700 & 58 & 1.9996554851532 & 0.7516820430756 \\
2800 & 46 & 1.7790718078613 & 1.3243298530579 \\
2900 & 90 & 3.4239289760590 & 0.1136860847473 \\
3000 & 64 & 2.7354006767273 & 1.8070988655090 \\
3100 & 75 & 3.0655438899994 & 1.6953222751617 \\
3200 & 53 & 2.4814045429230 & 2.6817638874054 \\
3300 & 52 & 2.3263633251190 & 3.4485423564911 \\
3400 & 71 & 3.7041838169098 & 2.5708866119385 \\
3500 & 75 & 3.6399040222168 & 4.0877647399902 \\
3600 & 101 & 5.5140559673309 & 5.1446208953857 \\
3700 & 77 & 4.0308992862701 & 4.3328595161438 \\
3800 & 97 & 5.7853453159332 & 3.8730812072754 \\
3900 & 72 & 4.2248957157135 & 9.8103401660919 \\
4000 & 136 & 8.5632438659668 & 6.2643780708313 \\
4100 & 147 & 10.1884658336639 & 1.3673396110535 \\
4200 & 70 & 5.2402031421661 & 6.8190603256226 \\
4300 & 98 & 7.2523105144501 & 5.8085365295410 \\
4400 & 53 & 3.8044888973236 & 10.1782982349396 \\
4500 & 133 & 9.8222827911377 & 5.1262011528015 \\
4600 & 237 & 19.1437103748322 & -3.3280611038208 \\
4700 & 62 & 4.8871183395386 & 12.3196654319763 \\
4800 & 144 & 12.4125111103058 & 5.6006090641022 \\
4900 & 57 & 4.7494163513184 & 14.3763835430145 \\
5000 & 63 & 5.8028347492218 & 14.4247493743896 \\
5100 & 39 & 3.4337244033813 & 18.8006572723389 \\
\bottomrule
\end{longtable}

\begin{longtable}{lrrr}

\caption{Результаты работы метода Якоби с компиляцией}\\
\toprule
 & Количество итераций & Затраченное время & Разность времени \\
Порядок матрицы &  &  &  \\
\midrule
\endfirsthead
\toprule
 & Количество итераций & Затраченное время & Разность времени \\
Порядок матрицы &  &  &  \\
\midrule
\endhead
\midrule
\midrule
\endfoot
\bottomrule
\endlastfoot
100 & 52 & 1.2462661266327 & -1.2436404228210 \\
200 & 78 & 0.0846045017242 & -0.0828149318695 \\
300 & 130 & 0.1750047206879 & -0.1708397865295 \\
400 & 76 & 0.1483836174011 & -0.1358969211578 \\
500 & 126 & 0.2856369018555 & -0.2615244388580 \\
600 & 99 & 0.2789611816406 & -0.2436184883118 \\
700 & 97 & 0.3174338340759 & -0.2686955928802 \\
800 & 69 & 0.2538394927979 & -0.1808216571808 \\
900 & 69 & 0.2932667732239 & -0.1869778633118 \\
1000 & 44 & 0.2109968662262 & -0.0719695091248 \\
1100 & 80 & 0.4322943687439 & -0.2217550277710 \\
1200 & 67 & 0.3959450721741 & -0.1570396423340 \\
1300 & 142 & 0.9175875186920 & -0.6101219654083 \\
1400 & 59 & 0.4157123565674 & -0.0226261615753 \\
1500 & 99 & 0.7937390804291 & -0.2969167232513 \\
1600 & 44 & 0.3854153156281 & 0.1953899860382 \\
1700 & 96 & 0.8642973899841 & -0.1837100982666 \\
1800 & 63 & 0.6015193462372 & 0.2112958431244 \\
1900 & 77 & 0.7820498943329 & 0.1739504337311 \\
2000 & 78 & 0.8397419452667 & 0.4113516807556 \\
2100 & 51 & 0.5938351154327 & 0.7683730125427 \\
2200 & 96 & 1.1946210861206 & 0.2949419021606 \\
2300 & 66 & 0.9122271537781 & 0.8047914505005 \\
2400 & 80 & 1.1049659252167 & 0.7982845306396 \\
2500 & 56 & 0.8440186977386 & 1.3180229663849 \\
2600 & 104 & 1.6135838031769 & 0.8111138343811 \\
2700 & 58 & 0.9647972583771 & 1.7865402698517 \\
2800 & 46 & 0.8365287780762 & 2.2668728828430 \\
2900 & 90 & 1.8695499897003 & 1.6680650711060 \\
3000 & 64 & 1.4069402217865 & 3.1355593204498 \\
3100 & 75 & 1.7986004352570 & 2.9622657299042 \\
3200 & 53 & 1.3548674583435 & 3.8083009719849 \\
3300 & 52 & 1.2982103824615 & 4.4766952991486 \\
3400 & 71 & 1.8741035461426 & 4.4009668827057 \\
3500 & 75 & 2.0641407966614 & 5.6635279655457 \\
3600 & 101 & 2.8348221778870 & 7.8238546848297 \\
3700 & 77 & 2.2926752567291 & 6.0710835456848 \\
3800 & 97 & 2.9539735317230 & 6.7044529914856 \\
3900 & 72 & 2.2821326255798 & 11.7531032562256 \\
4000 & 136 & 4.3838644027710 & 10.4437575340271 \\
4100 & 147 & 5.0060496330261 & 6.5497558116913 \\
4200 & 70 & 2.4042267799377 & 9.6550366878510 \\
4300 & 98 & 3.6198911666870 & 9.4409558773041 \\
4400 & 53 & 1.9632833003998 & 12.0195038318634 \\
4500 & 133 & 5.0482232570648 & 9.9002606868744 \\
4600 & 237 & 9.3684167861938 & 6.4472324848175 \\
4700 & 62 & 2.6154508590698 & 14.5913329124451 \\
4800 & 144 & 5.9540500640869 & 12.0590701103210 \\
4900 & 57 & 2.4468579292297 & 16.6789419651031 \\
5000 & 63 & 2.8342251777649 & 17.3933589458466 \\
5100 & 39 & 1.7981827259064 & 20.4361989498138 \\
\end{longtable}

\begin{longtable}{lrrr}
\caption{Результаты работы метода Гаусса-Зейделя без компиляции}\\
\toprule
 & Количество итераций & Затраченное время & Разность времени \\
Порядок матрицы &  &  &  \\
\midrule
\endfirsthead
\toprule
 & Количество итераций & Затраченное время & Разность времени \\
Порядок матрицы &  &  &  \\
\midrule
\endhead
\midrule
\midrule
\endfoot
\bottomrule
\endlastfoot
100 & 11 & 0.0193138122559 & -0.0166881084442 \\
200 & 13 & 0.0280597209930 & -0.0262701511383 \\
300 & 11 & 0.0442860126495 & -0.0401210784912 \\
400 & 8 & 0.0369372367859 & -0.0244505405426 \\
500 & 14 & 0.0744934082031 & -0.0503809452057 \\
600 & 16 & 0.1111514568329 & -0.0758087635040 \\
700 & 10 & 0.0775518417358 & -0.0288136005402 \\
800 & 14 & 0.1128537654877 & -0.0398359298706 \\
900 & 8 & 0.0812444686890 & 0.0250444412231 \\
1000 & 12 & 0.1285607814789 & 0.0104665756226 \\
1100 & 13 & 0.1541528701782 & 0.0563864707947 \\
1200 & 11 & 0.1443464756012 & 0.0945589542389 \\
1300 & 15 & 0.2733049392700 & 0.0341606140137 \\
1400 & 12 & 0.2351272106171 & 0.1579589843750 \\
1500 & 13 & 0.2502281665802 & 0.2465941905975 \\
1600 & 13 & 0.2620465755463 & 0.3187587261200 \\
1700 & 12 & 0.2395026683807 & 0.4410846233368 \\
1800 & 9 & 0.1971158981323 & 0.6156992912292 \\
1900 & 9 & 0.2035489082336 & 0.7524514198303 \\
2000 & 14 & 0.3440911769867 & 0.9070024490356 \\
2100 & 10 & 0.2535443305969 & 1.1086637973785 \\
2200 & 11 & 0.3078474998474 & 1.1817154884338 \\
2300 & 10 & 0.2965457439423 & 1.4204728603363 \\
2400 & 10 & 0.3074960708618 & 1.5957543849945 \\
2500 & 16 & 0.4939227104187 & 1.6681189537048 \\
2600 & 9 & 0.3807876110077 & 2.0439100265503 \\
2700 & 11 & 0.4530491828918 & 2.2982883453369 \\
2800 & 13 & 0.5149223804474 & 2.5884792804718 \\
2900 & 15 & 0.6519904136658 & 2.8856246471405 \\
3000 & 10 & 0.4015343189240 & 4.1409652233124 \\
3100 & 15 & 0.6651651859283 & 4.0957009792328 \\
3200 & 10 & 0.4945995807648 & 4.6685688495636 \\
3300 & 10 & 0.5265166759491 & 5.2483890056610 \\
3400 & 15 & 0.8306963443756 & 5.4443740844727 \\
3500 & 14 & 0.7629737854004 & 6.9646949768066 \\
3600 & 15 & 0.8384444713593 & 9.8202323913574 \\
3700 & 10 & 0.6754627227783 & 7.6882960796356 \\
3800 & 12 & 0.7503695487976 & 8.9080569744110 \\
3900 & 9 & 0.5795228481293 & 13.4557130336761 \\
4000 & 13 & 0.8064048290253 & 14.0212171077728 \\
4100 & 12 & 0.9270818233490 & 10.6287236213684 \\
4200 & 9 & 0.6576917171478 & 11.4015717506409 \\
4300 & 10 & 0.7700955867767 & 12.2907514572144 \\
4400 & 9 & 0.7986357212067 & 13.1841514110565 \\
4500 & 11 & 0.9651329517365 & 13.9833509922028 \\
4600 & 11 & 1.0155260562897 & 14.8001232147217 \\
4700 & 10 & 0.9815475940704 & 16.2252361774445 \\
4800 & 12 & 1.1166610717773 & 16.8964591026306 \\
4900 & 12 & 1.4609208106995 & 17.6648790836334 \\
5000 & 11 & 1.6296567916870 & 18.5979273319244 \\
5100 & 10 & 1.7612597942352 & 20.4731218814850 \\
\end{longtable}

\begin{longtable}{lrrr}
\caption{Результаты работы метода Гаусса-Зейделя с компиляцией}\\
\toprule
 & Количество итераций & Затраченное время & Разность времени \\
Порядок матрицы &  &  &  \\
\midrule
\endfirsthead
\toprule
 & Количество итераций & Затраченное время & Разность времени \\
Порядок матрицы &  &  &  \\
\midrule
\endhead
\midrule
\midrule
\endfoot
\bottomrule
\endlastfoot
100 & 11 & 0.8528187274933 & -0.8501930236816 \\
200 & 13 & 0.0114216804504 & -0.0096321105957 \\
300 & 11 & 0.0152025222778 & -0.0110375881195 \\
400 & 8 & 0.0139842033386 & -0.0014975070953 \\
500 & 14 & 0.0313191413879 & -0.0072066783905 \\
600 & 16 & 0.0441782474518 & -0.0088355541229 \\
700 & 10 & 0.0334949493408 & 0.0152432918549 \\
800 & 14 & 0.0524339675903 & 0.0205838680267 \\
900 & 8 & 0.0387454032898 & 0.0675435066223 \\
1000 & 12 & 0.0553576946259 & 0.0836696624756 \\
1100 & 13 & 0.0708754062653 & 0.1396639347076 \\
1200 & 11 & 0.0667796134949 & 0.1721258163452 \\
1300 & 15 & 0.0950477123260 & 0.2124178409576 \\
1400 & 12 & 0.0836653709412 & 0.3094208240509 \\
1500 & 13 & 0.0950589179993 & 0.4017634391785 \\
1600 & 13 & 0.1050190925598 & 0.4757862091064 \\
1700 & 12 & 0.1085293292999 & 0.5720579624176 \\
1800 & 9 & 0.0917103290558 & 0.7211048603058 \\
1900 & 9 & 0.0961143970490 & 0.8598859310150 \\
2000 & 14 & 0.1583130359650 & 1.0927805900574 \\
2100 & 10 & 0.1257097721100 & 1.2364983558655 \\
2200 & 11 & 0.1329574584961 & 1.3566055297852 \\
2300 & 10 & 0.1396324634552 & 1.5773861408234 \\
2400 & 10 & 0.1433229446411 & 1.7599275112152 \\
2500 & 16 & 0.2383604049683 & 1.9236812591553 \\
2600 & 9 & 0.1377286911011 & 2.2869689464569 \\
2700 & 11 & 0.1775202751160 & 2.5738172531128 \\
2800 & 13 & 0.2212841510773 & 2.8821175098419 \\
2900 & 15 & 0.2562315464020 & 3.2813835144043 \\
3000 & 10 & 0.1712708473206 & 4.3712286949158 \\
3100 & 15 & 0.2680990695953 & 4.4927670955658 \\
3200 & 10 & 0.1988301277161 & 4.9643383026123 \\
3300 & 10 & 0.2039537429810 & 5.5709519386292 \\
3400 & 15 & 0.3246674537659 & 5.9504029750824 \\
3500 & 14 & 0.3032133579254 & 7.4244554042816 \\
3600 & 15 & 0.3314881324768 & 10.3271887302399 \\
3700 & 10 & 0.2457180023193 & 8.1180408000946 \\
3800 & 12 & 0.2866723537445 & 9.3717541694641 \\
3900 & 9 & 0.2322742938995 & 13.8029615879059 \\
4000 & 13 & 0.3418791294098 & 14.4857428073883 \\
4100 & 12 & 0.3655488491058 & 11.1902565956116 \\
4200 & 9 & 0.2557041645050 & 11.8035593032837 \\
4300 & 10 & 0.2934572696686 & 12.7673897743225 \\
4400 & 9 & 0.2666175365448 & 13.7161695957184 \\
4500 & 11 & 0.3482410907745 & 14.6002428531647 \\
4600 & 11 & 0.3571336269379 & 15.4585156440735 \\
4700 & 10 & 0.3390650749207 & 16.8677186965942 \\
4800 & 12 & 0.4145016670227 & 17.5986185073853 \\
4900 & 12 & 0.4452347755432 & 18.6805651187897 \\
5000 & 11 & 0.4046220779419 & 19.8229620456696 \\
5100 & 10 & 0.3793203830719 & 21.8550612926483 \\
\end{longtable}

\begin{longtable}{lrrr}
\caption{Результаты работы метода релаксации с параметром $\omega=0.3$ без компиляции}\\
\toprule
 & Количество итераций & Затраченное время & Разность времени \\
Порядок матрицы &  &  &  \\
\midrule
\endfirsthead
\toprule
 & Количество итераций & Затраченное время & Разность времени \\
Порядок матрицы &  &  &  \\
\midrule
\endhead
\midrule
\midrule
\endfoot
\bottomrule
\endlastfoot
100 & 69 & 0.1005275249481 & -0.0979018211365 \\
200 & 57 & 0.1206607818604 & -0.1188712120056 \\
300 & 49 & 0.1425139904022 & -0.1383490562439 \\
400 & 56 & 0.2053191661835 & -0.1928324699402 \\
500 & 65 & 0.3115594387054 & -0.2874469757080 \\
600 & 63 & 0.4179477691650 & -0.3826050758362 \\
700 & 58 & 0.4500634670258 & -0.4013252258301 \\
800 & 63 & 0.5144131183624 & -0.4413952827454 \\
900 & 56 & 0.5079410076141 & -0.4016520977020 \\
1000 & 52 & 0.5479519367218 & -0.4089245796204 \\
1100 & 52 & 0.5866100788116 & -0.3760707378387 \\
1200 & 60 & 0.7877483367920 & -0.5488429069519 \\
1300 & 51 & 0.8253550529480 & -0.5178894996643 \\
1400 & 68 & 1.0837075710297 & -0.6906213760376 \\
1500 & 61 & 1.0677292346954 & -0.5709068775177 \\
1600 & 64 & 1.1952900886536 & -0.6144847869873 \\
1700 & 57 & 1.1720876693726 & -0.4915003776550 \\
1800 & 47 & 1.0376000404358 & -0.2247848510742 \\
1900 & 50 & 1.1995959281921 & -0.2435956001282 \\
2000 & 61 & 1.4129948616028 & -0.1619012355804 \\
2100 & 60 & 1.6161978244781 & -0.2539896965027 \\
2200 & 72 & 2.0117256641388 & -0.5221626758575 \\
2300 & 57 & 1.7831284999847 & -0.0661098957062 \\
2400 & 58 & 1.8348162174225 & 0.0684342384338 \\
2500 & 58 & 1.9607927799225 & 0.2012488842010 \\
2600 & 49 & 1.7928290367126 & 0.6318686008453 \\
2700 & 46 & 1.7378003597260 & 1.0135371685028 \\
2800 & 49 & 1.9201891422272 & 1.1832125186920 \\
2900 & 59 & 2.3988935947418 & 1.1387214660645 \\
3000 & 66 & 2.7753820419312 & 1.7671175003052 \\
3100 & 61 & 2.5413675308228 & 2.2194986343384 \\
3200 & 45 & 1.8975560665131 & 3.2656123638153 \\
3300 & 46 & 2.2226252555847 & 3.5522804260254 \\
3400 & 47 & 2.4044091701508 & 3.8706612586975 \\
3500 & 52 & 2.7252712249756 & 5.0023975372314 \\
3600 & 64 & 3.5677058696747 & 7.0909709930420 \\
3700 & 46 & 2.6802191734314 & 5.6835396289825 \\
3800 & 55 & 3.2846119403839 & 6.3738145828247 \\
3900 & 59 & 3.7781403064728 & 10.2570955753326 \\
4000 & 74 & 4.7115051746368 & 10.1161167621613 \\
4100 & 66 & 4.5059747695923 & 7.0498306751251 \\
4200 & 48 & 3.2124037742615 & 8.8468596935272 \\
4300 & 54 & 3.9817228317261 & 9.0791242122650 \\
4400 & 52 & 3.9754226207733 & 10.0073645114899 \\
4500 & 53 & 4.1944501399994 & 10.7540338039398 \\
4600 & 47 & 3.5656802654266 & 12.2499690055847 \\
4700 & 64 & 5.0644659996033 & 12.1423177719116 \\
4800 & 68 & 5.4379923343658 & 12.5751278400421 \\
4900 & 49 & 4.2294104099274 & 14.8963894844055 \\
5000 & 50 & 4.4580690860748 & 15.7695150375366 \\
5100 & 68 & 5.9131090641022 & 16.3212726116180 \\
\end{longtable}

\begin{longtable}{lrrr}
\caption{Результаты работы метода релаксации с параметром $\omega=0.3$ с компиляцией}\\
\toprule
 & Количество итераций & Затраченное время & Разность времени \\
Порядок матрицы &  &  &  \\
\midrule
\endfirsthead
\toprule
 & Количество итераций & Затраченное время & Разность времени \\
Порядок матрицы &  &  &  \\
\midrule
\endhead
\midrule
\midrule
\endfoot
\bottomrule
\endlastfoot
100 & 69 & 0.0599658489227 & -0.0573401451111 \\
200 & 57 & 0.0528619289398 & -0.0510723590851 \\
300 & 49 & 0.0694601535797 & -0.0652952194214 \\
400 & 56 & 0.1081957817078 & -0.0957090854645 \\
500 & 65 & 0.1574335098267 & -0.1333210468292 \\
600 & 63 & 0.1857311725616 & -0.1503884792328 \\
700 & 58 & 0.2014629840851 & -0.1527247428894 \\
800 & 63 & 0.2478799819946 & -0.1748621463776 \\
900 & 56 & 0.3290808200836 & -0.2227919101715 \\
1000 & 52 & 0.3150672912598 & -0.1760399341583 \\
1100 & 52 & 0.3454356193542 & -0.1348962783813 \\
1200 & 60 & 0.4408962726593 & -0.2019908428192 \\
1300 & 51 & 0.3937158584595 & -0.0862503051758 \\
1400 & 68 & 0.5974662303925 & -0.2043800354004 \\
1500 & 61 & 0.5677540302277 & -0.0709316730499 \\
1600 & 64 & 0.6298723220825 & -0.0490670204163 \\
1700 & 57 & 0.6181175708771 & 0.0624697208405 \\
1800 & 47 & 0.5932829380035 & 0.2195322513580 \\
1900 & 50 & 0.6457016468048 & 0.3102986812592 \\
2000 & 61 & 0.7719609737396 & 0.4791326522827 \\
2100 & 60 & 0.8635165691376 & 0.4986915588379 \\
2200 & 72 & 1.0985848903656 & 0.3909780979156 \\
2300 & 57 & 0.9858880043030 & 0.7311305999756 \\
2400 & 58 & 1.0942242145538 & 0.8090262413025 \\
2500 & 58 & 1.0667393207550 & 1.0953023433685 \\
2600 & 49 & 0.9036757946014 & 1.5210218429565 \\
2700 & 46 & 0.7348537445068 & 2.0164837837219 \\
2800 & 49 & 0.9248378276825 & 2.1785638332367 \\
2900 & 59 & 1.4107966423035 & 2.1268184185028 \\
3000 & 66 & 1.6211378574371 & 2.9213616847992 \\
3100 & 61 & 1.1287012100220 & 3.6321649551392 \\
3200 & 45 & 0.9213261604309 & 4.2418422698975 \\
3300 & 46 & 1.1303019523621 & 4.6446037292480 \\
3400 & 47 & 1.2875993251801 & 4.9874711036682 \\
3500 & 52 & 1.6717109680176 & 6.0559577941895 \\
3600 & 64 & 1.9996886253357 & 8.6589882373810 \\
3700 & 46 & 1.2649641036987 & 7.0987946987152 \\
3800 & 55 & 1.4309070110321 & 8.2275195121765 \\
3900 & 59 & 1.5354316234589 & 12.4998042583466 \\
4000 & 74 & 1.9188091754913 & 12.9088127613068 \\
4100 & 66 & 1.8247318267822 & 9.7310736179352 \\
4200 & 48 & 1.3616037368774 & 10.6976597309113 \\
4300 & 54 & 1.5901432037354 & 11.4707038402557 \\
4400 & 52 & 1.5206227302551 & 12.4621644020081 \\
4500 & 53 & 1.6625690460205 & 13.2859148979187 \\
4600 & 47 & 1.5783286094666 & 14.2373206615448 \\
4700 & 64 & 2.3678333759308 & 14.8389503955841 \\
4800 & 68 & 2.5054790973663 & 15.5076410770416 \\
4900 & 49 & 1.9110627174377 & 17.2147371768951 \\
5000 & 50 & 2.0753562450409 & 18.1522278785706 \\
5100 & 68 & 2.9693207740784 & 19.2650609016418 \\
\end{longtable}

\begin{longtable}{lrrr}
\caption{Результаты работы метода релаксации с параметром $\omega=0.5$ без компиляции}\\
\toprule
 & Количество итераций & Затраченное время & Разность времени \\
Порядок матрицы &  &  &  \\
\midrule
\endfirsthead
\toprule
 & Количество итераций & Затраченное время & Разность времени \\
Порядок матрицы &  &  &  \\
\midrule
\endhead
\midrule
\midrule
\endfoot
\bottomrule
\endlastfoot
100 & 38 & 0.0402333736420 & -0.0376076698303 \\
200 & 33 & 0.0642549991608 & -0.0624654293060 \\
300 & 26 & 0.0776267051697 & -0.0734617710114 \\
400 & 30 & 0.1137201786041 & -0.1012334823608 \\
500 & 35 & 0.1637310981750 & -0.1396186351776 \\
600 & 34 & 0.1938447952271 & -0.1585021018982 \\
700 & 33 & 0.2255289554596 & -0.1767907142639 \\
800 & 36 & 0.3112046718597 & -0.2381868362427 \\
900 & 30 & 0.2800018787384 & -0.1737129688263 \\
1000 & 29 & 0.2887110710144 & -0.1496837139130 \\
1100 & 29 & 0.3312096595764 & -0.1206703186035 \\
1200 & 33 & 0.4665756225586 & -0.2276701927185 \\
1300 & 27 & 0.4497785568237 & -0.1423130035400 \\
1400 & 38 & 0.5659520626068 & -0.1728658676147 \\
1500 & 37 & 0.6763019561768 & -0.1794795989990 \\
1600 & 35 & 0.6617882251740 & -0.0809829235077 \\
1700 & 31 & 0.6256315708160 & 0.0549557209015 \\
1800 & 27 & 0.6776878833771 & 0.1351273059845 \\
1900 & 28 & 0.8466567993164 & 0.1093435287476 \\
2000 & 34 & 1.0954561233521 & 0.1556375026703 \\
2100 & 33 & 0.8126471042633 & 0.5495610237122 \\
2200 & 39 & 1.1553461551666 & 0.3342168331146 \\
2300 & 32 & 1.0061981678009 & 0.7108204364777 \\
2400 & 33 & 1.0012252330780 & 0.9020252227783 \\
2500 & 32 & 1.0747253894806 & 1.0873162746429 \\
2600 & 28 & 1.0089309215546 & 1.4157667160034 \\
2700 & 26 & 1.0140163898468 & 1.7373211383820 \\
2800 & 27 & 1.0354535579681 & 2.0679481029510 \\
2900 & 34 & 1.3720691204071 & 2.1655459403992 \\
3000 & 36 & 1.5658690929413 & 2.9766304492950 \\
3100 & 34 & 1.5124974250793 & 3.2483687400818 \\
3200 & 25 & 1.1553020477295 & 4.0078663825989 \\
3300 & 25 & 1.2062094211578 & 4.5686962604523 \\
3400 & 26 & 1.2996444702148 & 4.9754259586334 \\
3500 & 30 & 1.5593328475952 & 6.1683359146118 \\
3600 & 34 & 1.8481411933899 & 8.8105356693268 \\
3700 & 28 & 1.5744004249573 & 6.7893583774567 \\
3800 & 29 & 1.5425746440887 & 8.1158518791199 \\
3900 & 32 & 1.7468125820160 & 12.2884232997894 \\
4000 & 41 & 2.6003525257111 & 12.2272694110870 \\
4100 & 37 & 2.1964769363403 & 9.3593285083771 \\
4200 & 27 & 1.9353318214417 & 10.1239316463470 \\
4300 & 30 & 2.1774110794067 & 10.8834359645844 \\
4400 & 29 & 2.0164768695831 & 11.9663102626801 \\
4500 & 29 & 2.2515411376953 & 12.6969428062439 \\
4600 & 26 & 2.0696635246277 & 13.7459857463837 \\
4700 & 35 & 2.9880497455597 & 14.2187340259552 \\
4800 & 37 & 3.1538977622986 & 14.8592224121094 \\
4900 & 27 & 2.3380839824677 & 16.7877159118652 \\
5000 & 28 & 2.5390222072601 & 17.6885619163513 \\
5100 & 38 & 3.5170762538910 & 18.7173054218292 \\
\end{longtable}

\begin{longtable}{lrrr}
\caption{Результаты работы метода релаксации с параметром $\omega=0.5$ с компиляцией} \\
\toprule
 & Количество итераций & Затраченное время & Разность времени \\
Порядок матрицы &  &  &  \\
\midrule
\endfirsthead
\toprule
 & Количество итераций & Затраченное время & Разность времени \\
Порядок матрицы &  &  &  \\
\midrule
\endhead
\midrule
\midrule
\endfoot
\bottomrule
\endlastfoot
100 & 38 & 0.0248558521271 & -0.0222301483154 \\
200 & 33 & 0.0251035690308 & -0.0233139991760 \\
300 & 26 & 0.0362539291382 & -0.0320889949799 \\
400 & 30 & 0.0539436340332 & -0.0414569377899 \\
500 & 35 & 0.1520664691925 & -0.1279540061951 \\
600 & 34 & 0.1575365066528 & -0.1221938133240 \\
700 & 33 & 0.1297938823700 & -0.0810556411743 \\
800 & 36 & 0.1718380451202 & -0.0988202095032 \\
900 & 30 & 0.2172434329987 & -0.1109545230865 \\
1000 & 29 & 0.2170009613037 & -0.0779736042023 \\
1100 & 29 & 0.1729927062988 & 0.0375466346741 \\
1200 & 33 & 0.2178304195404 & 0.0210750102997 \\
1300 & 27 & 0.1742932796478 & 0.1331722736359 \\
1400 & 38 & 0.2733516693115 & 0.1197345256805 \\
1500 & 37 & 0.2846527099609 & 0.2121696472168 \\
1600 & 35 & 0.3049571514130 & 0.2758481502533 \\
1700 & 31 & 0.2814493179321 & 0.3991379737854 \\
1800 & 27 & 0.2532563209534 & 0.5595588684082 \\
1900 & 28 & 0.2941541671753 & 0.6618461608887 \\
2000 & 34 & 0.3561465740204 & 0.8949470520020 \\
2100 & 33 & 0.3768675327301 & 0.9853405952454 \\
2200 & 39 & 0.4861509799957 & 1.0034120082855 \\
2300 & 32 & 0.4221777915955 & 1.2948408126831 \\
2400 & 33 & 0.4465887546539 & 1.4566617012024 \\
2500 & 32 & 0.4615571498871 & 1.7004845142365 \\
2600 & 28 & 0.4367129802704 & 1.9879846572876 \\
2700 & 26 & 0.4402854442596 & 2.3110520839691 \\
2800 & 27 & 0.4604372978210 & 2.6429643630981 \\
2900 & 34 & 0.5776770114899 & 2.9599380493164 \\
3000 & 36 & 0.6508510112762 & 3.8916485309601 \\
3100 & 34 & 0.6246719360352 & 4.1361942291260 \\
3200 & 25 & 0.5412435531616 & 4.6219248771667 \\
3300 & 25 & 0.5218186378479 & 5.2530870437622 \\
3400 & 26 & 0.5811769962311 & 5.6938934326172 \\
3500 & 30 & 0.6890027523041 & 7.0386660099030 \\
3600 & 34 & 0.8069655895233 & 9.8517112731934 \\
3700 & 28 & 0.7306487560272 & 7.6331100463867 \\
3800 & 29 & 0.7524752616882 & 8.9059512615204 \\
3900 & 32 & 0.9109027385712 & 13.1243331432343 \\
4000 & 41 & 1.1246740818024 & 13.7029478549957 \\
4100 & 37 & 1.1245305538177 & 10.4312748908997 \\
4200 & 27 & 0.8497850894928 & 11.2094783782959 \\
4300 & 30 & 1.1448340415955 & 11.9160130023956 \\
4400 & 29 & 0.9561314582825 & 13.0266556739807 \\
4500 & 29 & 0.9992966651917 & 13.9491872787476 \\
4600 & 26 & 0.8902773857117 & 14.9253718852997 \\
4700 & 35 & 1.2723178863525 & 15.9344658851624 \\
4800 & 37 & 1.3840119838715 & 16.6291081905365 \\
4900 & 27 & 1.0260970592499 & 18.0997028350830 \\
5000 & 28 & 1.1025667190552 & 19.1250174045563 \\
5100 & 38 & 1.5428266525269 & 20.6915550231934 \\
\end{longtable}

\begin{longtable}{lrrr}
\caption{Результаты работы метода релаксации с параметром $\omega=1.2$ без компиляции}\\
\toprule
 & Количество итераций & Затраченное время & Разность времени \\
Порядок матрицы &  &  &  \\
\midrule
\endfirsthead
\toprule
 & Количество итераций & Затраченное время & Разность времени \\
Порядок матрицы &  &  &  \\
\midrule
\endhead
\midrule
\midrule
\endfoot
\bottomrule
\endlastfoot
100 & 12 & 0.0166332721710 & -0.0140075683594 \\
200 & 15 & 0.0298423767090 & -0.0280528068542 \\
300 & 15 & 0.0451591014862 & -0.0409941673279 \\
400 & 21 & 0.0882899761200 & -0.0758032798767 \\
500 & 17 & 0.0889542102814 & -0.0648417472839 \\
600 & 25 & 0.1537625789642 & -0.1184198856354 \\
700 & 12 & 0.0823657512665 & -0.0336275100708 \\
800 & 22 & 0.1785562038422 & -0.1055383682251 \\
900 & 14 & 0.1360051631927 & -0.0297162532806 \\
1000 & 17 & 0.2198336124420 & -0.0808062553406 \\
1100 & 19 & 0.2363474369049 & -0.0258080959320 \\
1200 & 15 & 0.2095620632172 & 0.0293433666229 \\
1300 & 25 & 0.3633921146393 & -0.0559265613556 \\
1400 & 13 & 0.2153842449188 & 0.1777019500732 \\
1500 & 12 & 0.2089300155640 & 0.2878923416138 \\
1600 & 16 & 0.3037371635437 & 0.2770681381226 \\
1700 & 17 & 0.3443384170532 & 0.3362488746643 \\
1800 & 14 & 0.3023431301117 & 0.5104720592499 \\
1900 & 13 & 0.3030085563660 & 0.6529917716980 \\
2000 & 19 & 0.4842064380646 & 0.7668871879578 \\
2100 & 13 & 0.3193805217743 & 1.0428276062012 \\
2200 & 16 & 0.4808347225189 & 1.0087282657623 \\
2300 & 12 & 0.3885960578918 & 1.3284225463867 \\
2400 & 14 & 0.4100711345673 & 1.4931793212891 \\
2500 & 23 & 0.8611750602722 & 1.3008666038513 \\
2600 & 12 & 0.4651086330414 & 1.9595890045166 \\
2700 & 14 & 0.5512661933899 & 2.2000713348389 \\
2800 & 18 & 0.7222700119019 & 2.3811316490173 \\
2900 & 21 & 0.9656829833984 & 2.5719320774078 \\
3000 & 13 & 0.5615544319153 & 3.9809451103210 \\
3100 & 25 & 1.1021821498871 & 3.6586840152740 \\
3200 & 14 & 0.6591060161591 & 4.5040624141693 \\
3300 & 15 & 0.7905263900757 & 4.9843792915344 \\
3400 & 23 & 1.2351336479187 & 5.0399367809296 \\
3500 & 19 & 1.0570447444916 & 6.6706240177155 \\
3600 & 26 & 1.4765775203705 & 9.1820993423462 \\
3700 & 15 & 0.7902989387512 & 7.5734598636627 \\
3800 & 16 & 0.8549127578735 & 8.8035137653351 \\
3900 & 12 & 0.7487325668335 & 13.2865033149719 \\
4000 & 16 & 1.0110464096069 & 13.8165755271912 \\
4100 & 14 & 0.9295926094055 & 10.6262128353119 \\
4200 & 12 & 0.7497112751007 & 11.3095521926880 \\
4300 & 16 & 1.1362178325653 & 11.9246292114258 \\
4400 & 12 & 0.8932015895844 & 13.0895855426788 \\
4500 & 14 & 1.1126706600189 & 13.8358132839203 \\
4600 & 15 & 1.2443811893463 & 14.5712680816650 \\
4700 & 16 & 1.3211269378662 & 15.8856568336487 \\
4800 & 17 & 1.4305002689362 & 16.5826199054718 \\
4900 & 20 & 1.7546513080597 & 17.3711485862732 \\
5000 & 18 & 1.5322151184082 & 18.6953690052032 \\
5100 & 12 & 1.1090400218964 & 21.1253416538239 \\
\end{longtable}

\begin{longtable}{lrrr}
\label{d10}\\
\caption{Результаты работы метода релаксации с параметром $\omega=1.2$ с компиляцией}\\
\toprule
 & Количество итераций & Затраченное время & Разность времени \\
Порядок матрицы &  &  &  \\
\midrule
\endfirsthead
\toprule
 & Количество итераций & Затраченное время & Разность времени \\
Порядок матрицы &  &  &  \\
\midrule
\endhead
\midrule
\midrule
\endfoot
\bottomrule
\endlastfoot
100 & 12 & 0.0082290172577 & -0.0056033134460 \\
200 & 15 & 0.0175986289978 & -0.0158090591431 \\
300 & 15 & 0.0155405998230 & -0.0113756656647 \\
400 & 21 & 0.0392782688141 & -0.0267915725708 \\
500 & 17 & 0.0374767780304 & -0.0133643150330 \\
600 & 25 & 0.0688302516937 & -0.0334875583649 \\
700 & 12 & 0.0399148464203 & 0.0088233947754 \\
800 & 22 & 0.0844564437866 & -0.0114386081696 \\
900 & 14 & 0.0656771659851 & 0.0406117439270 \\
1000 & 17 & 0.0807631015778 & 0.0582642555237 \\
1100 & 19 & 0.1066310405731 & 0.1039083003998 \\
1200 & 15 & 0.0918235778809 & 0.1470818519592 \\
1300 & 25 & 0.1712594032288 & 0.1362061500549 \\
1400 & 13 & 0.0905311107635 & 0.3025550842285 \\
1500 & 12 & 0.1010007858276 & 0.3958215713501 \\
1600 & 16 & 0.1719157695770 & 0.4088895320892 \\
1700 & 17 & 0.2272701263428 & 0.4533171653748 \\
1800 & 14 & 0.1994137763977 & 0.6134014129639 \\
1900 & 13 & 0.1798992156982 & 0.7761011123657 \\
2000 & 19 & 0.2744848728180 & 0.9766087532043 \\
2100 & 13 & 0.3218951225281 & 1.0403130054474 \\
2200 & 16 & 0.3746881484985 & 1.1148748397827 \\
2300 & 12 & 0.3302288055420 & 1.3867897987366 \\
2400 & 14 & 0.3215222358704 & 1.5817282199860 \\
2500 & 23 & 0.4818553924561 & 1.6801862716675 \\
2600 & 12 & 0.2393367290497 & 2.1853609085083 \\
2700 & 14 & 0.2901587486267 & 2.4611787796021 \\
2800 & 18 & 0.3887481689453 & 2.7146534919739 \\
2900 & 21 & 0.4525003433228 & 3.0851147174835 \\
3000 & 13 & 0.2949171066284 & 4.2475824356079 \\
3100 & 25 & 0.5747072696686 & 4.1861588954926 \\
3200 & 14 & 0.3570778369904 & 4.8060905933380 \\
3300 & 15 & 0.3736119270325 & 5.4012937545776 \\
3400 & 23 & 0.5868144035339 & 5.6882560253143 \\
3500 & 19 & 0.5423560142517 & 7.1853127479553 \\
3600 & 26 & 0.7527976036072 & 9.9058792591095 \\
3700 & 15 & 0.4427814483643 & 7.9209773540497 \\
3800 & 16 & 0.5006721019745 & 9.1577544212341 \\
3900 & 12 & 0.3766620159149 & 13.6585738658905 \\
4000 & 16 & 0.5179705619812 & 14.3096513748169 \\
4100 & 14 & 0.4637806415558 & 11.0920248031616 \\
4200 & 12 & 0.4140400886536 & 11.6452233791351 \\
4300 & 16 & 0.5790660381317 & 12.4817810058594 \\
4400 & 12 & 0.4403429031372 & 13.5424442291260 \\
4500 & 14 & 0.5270836353302 & 14.4214003086090 \\
4600 & 15 & 0.6532554626465 & 15.1623938083649 \\
4700 & 16 & 0.6440670490265 & 16.5627167224884 \\
4800 & 17 & 0.7211799621582 & 17.2919402122498 \\
4900 & 20 & 0.8769676685333 & 18.2488322257996 \\
5000 & 18 & 0.7813432216644 & 19.4462409019470 \\
5100 & 12 & 0.5561645030975 & 21.6782171726227 \\
\end{longtable}

\subsubsection{Сравнение точности}
Сравнение точности будет производиться в два этапа на примере СЛАУ с матрицей 
порядка $2000$. На первом этапе ищется невязка решения.
На втором сравнивается относительная погрешность решения
с относительной погрешностью возмущенной системы,
то есть системы, полученной путем прибавления  $0.001$ к каждому коэффициенту 
каждого уравнения включая свободный член.

Как можно заметить из таблиц 
\ref{res} и \ref{errs}, самым точным является метод Гаусса-Зейделя.
\begin{longtable}{llr}
\caption{Нормы невязки} \label{res} \\
\toprule
 & Метод & Норма невязки \\
\midrule
\endfirsthead
\caption[]{Нормы невязки} \\
\toprule
 & Метод & Норма невязки \\
\midrule
\endhead
\midrule
\multicolumn{3}{r}{Continued on next page} \\
\midrule
\endfoot
\bottomrule
\endlastfoot
0 & Якоби & 1.2763584878147 \\
1 & Гаусса-Зейделя & 0.0306131739312 \\
2 & Релаксации $\omega=0.3$ & 4.2002312764003 \\
3 & Релаксации $\omega = 0.5$ & 2.1604302269335 \\
4 & Релаксации $\omega = 1.2$ & 0.1404159486946 \\
\end{longtable}

\begin{longtable}{llrr}
\caption{Относительные погрешности} \label{errs} \\
\toprule
 & Метод & $\delta$ & $\delta$ после возмущения \\
\midrule
\endfirsthead
\caption[]{Относительные погрешности} \\
\toprule
 & Метод & $\delta$ & $\delta$ после возмущения \\
\midrule
\endhead
\midrule
\midrule
\endfoot
\bottomrule
\endlastfoot
0 & Якоби & 0.0000008073422 & 0.0000042597904 \\
1 & Гаусса-Зейделя & 0.0000006851802 & 0.0000043686549 \\
2 & Релаксации $\omega=0.3$ & 0.0000076816049 & 0.0000090249221 \\
3 & Релаксации $\omega = 0.5$ & 0.0000032535898 & 0.0000055289461 \\
4 & Релаксации $\omega = 1.2$ & 0.0000003172812 & 0.0000042732700 \\
\end{longtable}

\subsection{Сравнение алгоритмов решения СЛАУ с симметричной и положительно определенной матрицей}
Все сравнения будут производиться
на случайных симметричных
положительно определенных матрицах и матрицах Гилберта \cite{book} , которые являются плохо обусловленными.

Алгоритмы сравнения времени работы и точности  аналогичны алгоритмам использованным в пункте \ref{diag}.
\subsubsection{Сравнение времени работы алгоритмов на случайных симметричных положительно определенных матрицах}
Из таблиц \ref{gsSympos} -- \ref{rich} можно заметить,
что самым быстрым является метод Гаусса-Зейделя, метод Ричардсона является самым медленным.
\begin{longtable}{lrrr}
\label{gsSympos}\\
\caption{Результаты  работы метода Гаусса-Зейделя}\\
\toprule
 & Количество итераций & Затраченное время & Разность времени \\
Порядок матрицы &  &  &  \\
\midrule
\endfirsthead
\toprule
 & Количество итераций & Затраченное время & Разность времени \\
Порядок матрицы &  &  &  \\
\midrule
\endhead
\midrule
\midrule
\endfoot
\bottomrule
\endlastfoot
100 & 401 & 2.3390643596649 & -2.3377883434296 \\
200 & 801 & 0.7135362625122 & -0.7120790481567 \\
300 & 1201 & 1.6115331649780 & -1.6070280075073 \\
400 & 1601 & 2.9337511062622 & -2.9233741760254 \\
500 & 1066 & 2.4235734939575 & -2.4038820266724 \\
600 & 2401 & 6.6838998794556 & -6.6506719589233 \\
700 & 560 & 1.8110439777374 & -1.7600324153900 \\
800 & 476 & 1.7363913059235 & -1.6598703861237 \\
900 & 720 & 3.0331983566284 & -2.9262795448303 \\
1000 & 331 & 1.6133313179016 & -1.4683439731598 \\
1100 & 185 & 0.9997541904449 & -0.8063507080078 \\
1200 & 192 & 1.1539945602417 & -0.9021136760712 \\
1300 & 251 & 1.6482443809509 & -1.3270256519318 \\
1400 & 248 & 1.7628011703491 & -1.3643929958344 \\
1500 & 111 & 0.8505480289459 & -0.3603699207306 \\
1600 & 65 & 0.5391702651978 & 0.0462150573730 \\
1700 & 102 & 0.9039313793182 & -0.2014229297638 \\
1800 & 64 & 0.6079511642456 & 0.2231085300446 \\
1900 & 63 & 0.6363825798035 & 0.3379044532776 \\
\end{longtable}

\begin{longtable}{lrrr}
\caption{Результаты работы метода релаксации с параметром $\omega=1.2$}\\
\toprule
 & Количество итераций & Затраченное время & Разность времени \\
Порядок матрицы &  &  &  \\
\midrule
\endfirsthead
\toprule
 & Количество итераций & Затраченное время & Разность времени \\
Порядок матрицы &  &  &  \\
\midrule
\endhead
\midrule
\midrule
\endfoot
\bottomrule
\endlastfoot
100 & 801 & 1.1301441192627 & -1.1288681030273 \\
200 & 1601 & 1.4322702884674 & -1.4308130741119 \\
300 & 2401 & 3.2345314025879 & -3.2300262451172 \\
400 & 3201 & 5.8470849990845 & -5.8367080688477 \\
500 & 2601 & 5.9274573326111 & -5.9077658653259 \\
600 & 4801 & 13.4231355190277 & -13.3899075984955 \\
700 & 1420 & 4.5338139533997 & -4.4828023910522 \\
800 & 2216 & 8.0146768093109 & -7.9381558895111 \\
900 & 1749 & 7.2374484539032 & -7.1305296421051 \\
1000 & 795 & 3.8509497642517 & -3.7059624195099 \\
1100 & 394 & 2.0995259284973 & -1.9061224460602 \\
1200 & 437 & 2.6578633785248 & -2.4059824943542 \\
1300 & 487 & 3.2416641712189 & -2.9204454421997 \\
1400 & 1257 & 9.1039156913757 & -8.7055075168610 \\
1500 & 392 & 3.0689554214478 & -2.5787773132324 \\
1600 & 140 & 1.1810905933380 & -0.5957052707672 \\
1700 & 290 & 2.6101050376892 & -1.9075965881348 \\
1800 & 216 & 2.0952432155609 & -1.2641835212708 \\
1900 & 159 & 1.6455583572388 & -0.6712713241577 \\
\end{longtable}

\begin{longtable}{lrrr}
\caption{Результат работы метода релаксации с параметром $\omega=1.5$}\\
\toprule
 & Количество итераций & Затраченное время & Разность времени \\
Порядок матрицы &  &  &  \\
\midrule
\endfirsthead
\toprule
 & Количество итераций & Затраченное время & Разность времени \\
Порядок матрицы &  &  &  \\
\midrule
\endhead
\midrule
\midrule
\endfoot
\bottomrule
\endlastfoot
100 & 801 & 0.3459229469299 & -0.3446469306946 \\
200 & 1601 & 1.4861764907837 & -1.4847192764282 \\
300 & 2401 & 3.4108848571777 & -3.4063796997070 \\
400 & 3201 & 6.0821325778961 & -6.0717556476593 \\
500 & 4001 & 9.6253893375397 & -9.6056978702545 \\
600 & 4801 & 13.9792003631592 & -13.9459724426270 \\
700 & 5601 & 18.9362955093384 & -18.8852839469910 \\
800 & 6401 & 24.0277609825134 & -23.9512400627136 \\
900 & 7201 & 29.8600468635559 & -29.7531280517578 \\
1000 & 4011 & 19.3795304298401 & -19.2345430850983 \\
1100 & 1915 & 10.3103036880493 & -10.1169002056122 \\
1200 & 1075 & 6.4404642581940 & -6.1885833740234 \\
1300 & 1473 & 9.8529915809631 & -9.5317728519440 \\
1400 & 5953 & 43.5349123477936 & -43.1365041732788 \\
1500 & 2323 & 17.9472532272339 & -17.4570751190186 \\
1600 & 862 & 7.2608704566956 & -6.6754851341248 \\
1700 & 768 & 6.8267784118652 & -6.1242699623108 \\
1800 & 1042 & 10.0620710849762 & -9.2310113906860 \\
1900 & 758 & 8.0394363403320 & -7.0651493072510 \\
\end{longtable}

\begin{longtable}{lrrr}
\label{rich}\\
\caption{Результат работы метода Ричардсона}\\
\toprule
 & Количество итераций & Затраченное время & Разность времени \\
Порядок матрицы &  &  &  \\
\midrule
\endfirsthead
\toprule
 & Количество итераций & Затраченное время & Разность времени \\
Порядок матрицы &  &  &  \\
\midrule
\endhead
\midrule
\midrule
\endfoot
\bottomrule
\endlastfoot
100 & 801 & 0.0877866744995 & -0.0874655246735 \\
200 & 1601 & 0.1379654407501 & -0.1364347934723 \\
300 & 2401 & 0.2654173374176 & -0.2606511116028 \\
400 & 3201 & 0.4632790088654 & -0.4527266025543 \\
500 & 4001 & 0.9214994907379 & -0.9019143581390 \\
600 & 4801 & 2.0674915313721 & -2.0337567329407 \\
700 & 5601 & 6.7622296810150 & -6.7098324298859 \\
800 & 6401 & 9.3341002464294 & -9.2570641040802 \\
900 & 7201 & 13.3334608078003 & -13.2260453701019 \\
1000 & 8001 & 15.0040786266327 & -14.8584368228912 \\
1100 & 8801 & 22.8945047855377 & -22.7024776935577 \\
1200 & 9601 & 27.4870173931122 & -27.2380068302155 \\
1300 & 10401 & 38.6106338500977 & -38.2916533946991 \\
1400 & 11201 & 46.4673750400543 & -46.0744259357452 \\
1500 & 12001 & 59.3388886451721 & -58.8576273918152 \\
1600 & 12801 & 65.2636866569519 & -64.6816787719727 \\
1700 & 13601 & 94.0645022392273 & -93.3679246902466 \\
1800 & 14401 & 107.8847098350525 & -107.0655002593994 \\
1900 & 15201 & 133.7427425384521 & -132.7753863334656 \\
2000 & 16001 & 243.1282382011414 & -242.0031116008759 \\
\end{longtable}

\pagebreak
\subsubsection{Сравнение точности}
Как можно заметить из таблиц \ref{res1} и \ref{errs1}, самым точным методом является метод Гаусса-Зейделя.
Точность заметно ниже, чем при решении СЛАУ с матрицей с диагональным преобладанием.

\begin{longtable}{llr}
\caption{Нормы невязки}
\label{res1}\\
\toprule
{} &                      Метод &   Норма невязки \\
\midrule
\endfirsthead
\caption[]{Нормы невязки} \\
\toprule
{} &                      Метод &   Норма невязки \\
\midrule
\endhead
\midrule
\multicolumn{3}{r}{{Continued on next page}} \\
\midrule
\endfoot

\bottomrule
\endlastfoot
0 &                 Ричардсона & 48771.498597651 \\
1 &            Гаусса-Зейделя & 28281.615624683 \\
2 &  Релаксации $\omega = 1.2$ & 38669.964384109 \\
3 &    Релаксации $\omega=1.5$ & 44615.878249346 \\
\end{longtable}

\begin{longtable}{llrr}
\caption{Относительные погрешности}
\label{errs1}\\
\toprule
{} &                      Метод &  $\delta$ & $\delta$ после возмущения \\
\midrule
\endfirsthead
\caption[]{Относительные погрешности} \\
\toprule
{} &                      Метод &  $\delta$ &  $\delta$ после возмущения \\
\midrule
\endhead
\midrule
\midrule
\endfoot

\bottomrule
\endlastfoot
0 &                 Ричардсона & 0.9997184 &                  0.9997184 \\
1 &            Гаусса-Зейлделя & 0.9915907 &                  0.9915907 \\
2 &  Релаксации $\omega = 1.2$ & 0.9767811 &                  0.9767811 \\
3 &    Релаксации $\omega=1.5$ & 0.9688871 &                  0.9688871 \\
\end{longtable}

\subsubsection{Сравнение времени работы алгоритмов на матрицах Гилберта}
Из таблиц \ref{gil1} -- \ref{rich_hilbert} можно заметить, что все методы работают на матрицах Гилберта очень медленно, но метод Гаусса-Зейделя быстрее остальных.
\begin{longtable}{lrrr}
\label{gil1}\\
\caption{Результат работы метода Гаусса-Зейделя для матриц Гилберта.}\\
\toprule
 & Количество итераций & Затраченное время & Разность времени \\
Порядок матрицы &  &  &  \\
\midrule
\endfirsthead
\toprule
 & Количество итераций & Затраченное время & Разность времени \\
Порядок матрицы &  &  &  \\
\midrule
\endhead
\midrule
\midrule
\endfoot
\bottomrule
\endlastfoot
100 & 401 & 1.9366397857666 & -1.9357037544250 \\
200 & 801 & 0.8523890972137 & -0.8469135761261 \\
300 & 1201 & 1.8368988037109 & -1.8289024829865 \\
400 & 1601 & 3.3037641048431 & -3.2885503768921 \\
500 & 2001 & 4.9675052165985 & -4.9415605068207 \\
600 & 2401 & 6.7978944778442 & -6.7548944950104 \\
700 & 2801 & 9.1904799938202 & -9.1319081783295 \\
800 & 3201 & 12.0941190719604 & -12.0100197792053 \\
900 & 3601 & 14.9489524364471 & -14.8329596519470 \\
\end{longtable}

\begin{longtable}{lrrr}
\caption{Результат работы метода релаксации с параметром $\omega=1.2$ для матриц Гилберта}\\
\toprule
 & Количество итераций & Затраченное время & Разность времени \\
Порядок матрицы &  &  &  \\
\midrule
\endfirsthead
\toprule
 & Количество итераций & Затраченное время & Разность времени \\
Порядок матрицы &  &  &  \\
\midrule
\endhead
\midrule
\endfoot
\bottomrule
\endlastfoot
100 & 801 & 1.0987691879272 & -1.0978331565857 \\
200 & 1601 & 1.5080630779266 & -1.5025875568390 \\
300 & 2401 & 3.4061763286591 & -3.3981800079346 \\
400 & 3201 & 6.1902682781219 & -6.1750545501709 \\
500 & 4001 & 9.5572087764740 & -9.5312640666962 \\
600 & 4801 & 14.1781008243561 & -14.1351008415222 \\
700 & 5601 & 18.7976863384247 & -18.7391145229340 \\
800 & 6401 & 24.5221281051636 & -24.4380288124084 \\
900 & 7201 & 30.8081021308899 & -30.6921093463898 \\
\end{longtable}

\begin{longtable}{lrrr}
\caption{Результат работы метода релаксации с параметром $\omega=1.5$ для матриц Гилберта}\\
\toprule
 & Количество итераций & Затраченное время & Разность времени \\
Порядок матрицы &  &  &  \\
\midrule
\endfirsthead
\toprule
 & Количество итераций & Затраченное время & Разность времени \\
Порядок матрицы &  &  &  \\
\midrule
\endhead
\midrule
\multicolumn{4}{r}{Continued on next page} \\
\midrule
\endfoot
\bottomrule
\endlastfoot
100 & 801 & 0.4468040466309 & -0.4458680152893 \\
200 & 1601 & 1.5727531909943 & -1.5672776699066 \\
300 & 2401 & 3.4723470211029 & -3.4643507003784 \\
400 & 3201 & 6.0610089302063 & -6.0457952022552 \\
500 & 4001 & 9.2504413127899 & -9.2244966030121 \\
600 & 4801 & 13.5998442173004 & -13.5568442344666 \\
700 & 5601 & 19.5377233028412 & -19.4791514873505 \\
800 & 6401 & 24.2146868705750 & -24.1305875778198 \\
900 & 7201 & 35.9259448051453 & -35.8099520206451 \\
\end{longtable}

\begin{longtable}{lrrr}
\label{rich_hilbert}\\
\caption{Результат работы метода Ричардсона для матриц Гилберта}\\
\toprule
 & Количество итераций & Затраченное время & Разность времени \\
Порядок матрицы &  &  &  \\
\midrule
\endfirsthead
\toprule
 & Количество итераций & Затраченное время & Разность времени \\
Порядок матрицы &  &  &  \\
\midrule
\endhead
\midrule
\endfoot
\bottomrule
\endlastfoot
100 & 801 & 0.0317203998566 & -0.0307843685150 \\
200 & 1601 & 0.2037823200226 & -0.1983067989349 \\
300 & 2401 & 0.6605954170227 & -0.6525990962982 \\
400 & 3201 & 1.6631236076355 & -1.6479098796844 \\
500 & 4001 & 3.4081640243530 & -3.3822193145752 \\
600 & 4801 & 6.8010828495026 & -6.7580828666687 \\
700 & 5601 & 10.9645364284515 & -10.9059646129608 \\
800 & 6401 & 16.6134688854218 & -16.5293695926666 \\
900 & 7201 & 34.6125929355621 & -34.4966001510620 \\
\end{longtable}

\subsubsection{Сравнение точности}
Матрицы Гилберта являются плохо обусловленными, что 
видно из таблиц \ref{resHilbert}, \ref{errsHilbert}. Поэтому итерационные методы СЛАУ не дали результата.
\begin{longtable}{llr}
\caption{Нормы невязки}
\label{resHilbert}\\
\toprule
{} &                      Метод &     Норма невязки \\
\midrule
\endfirsthead
\caption[]{Нормы невязки} \\
\toprule
{} &                      Метод &     Норма невязки \\
\midrule
\endhead
\midrule
\multicolumn{3}{r}{{Continued on next page}} \\
\midrule
\endfoot

\bottomrule
\endlastfoot
0 &                 Ричардсона & 1286682.770791929 \\
1 &            Гаусса-Зейлделя &  848369.191361920 \\
2 &  Релаксации $\omega = 1.2$ &  842002.442041366 \\
3 &    Релаксации $\omega=1.5$ &  854669.537678260 \\
\end{longtable}

\begin{longtable}{llrr}
\caption{Относительные погрешности}
\label{errsHilbert}\\
\toprule
{} &                      Метод &  $\delta$ &  $\delta$ после возмущения \\
\midrule
\endfirsthead
\caption[]{Относительные погрешности} \\
\toprule
{} &                      Метод &  $\delta$ &  $\delta$ после возмущения \\
\midrule
\endhead
\midrule
\midrule
\endfoot

\bottomrule
\endlastfoot
0 &                 Ричардсона & 1.0000000 &                  1.0000000 \\
1 &            Гаусса-Зейлделя & 1.0000000 &                  1.0000000 \\
2 &  Релаксации $\omega = 1.2$ & 1.0000000 &                  1.0000000 \\
3 &    Релаксации $\omega=1.5$ & 1.0000000 &                  1.0000000 \\
\end{longtable}

\pagebreak
\section*{ЗАКЛЮЧЕНИЕ}
    \addcontentsline{toc}{section}{ЗАКЛЮЧЕНИЕ}
    Все задачи курсовой работы были выполнены, но эту работу можно дополнить. Например:
    \begin{enumerate}
        \item переписать на более эффективный язык программирования, сравнить эффективность;
        \item создать более эффективный алгоритм генерации симметричных, положительно определенных матриц;
        \item провести расчеты на более производительном персональном компютере;
        \item провести расчеты на видеокарте, сравнить эффективность;
        \item реализовать другие итерационные методы решения СЛАУ.
    \end{enumerate}
    \pagebreak
    \printbibliography[title=БИБЛИОГРАФИЧЕСКИЙ СПИСОК]
\addcontentsline{toc}{section}{БИБЛИОГРАФИЧЕСКИЙ СПИСОК}
\end{document} 

